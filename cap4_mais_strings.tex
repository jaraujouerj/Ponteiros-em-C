\chapterimage{carbonated-water-device} % Chapter heading image

\chapter{Mais sobre Strings}
Bem, nós progredimos bastante em pouco tempo! Vamos recuar um pouco e ver o que foi feito no Capítulo 3 sobre cópia de strings, mas sob uma luz diferente. Considere a seguinte função:
\lstinputlisting[label=prog:meustrcpy4, caption= meu\_strcpy() com arrays, numbers=left, lastline=11]{code/prog4-1.c}

Lembre-se de que as strings são \textit{arrays} de caracteres. Aqui, escolhemos usar a notação de \textit{array} em vez da notação de ponteiro para fazer a cópia real. Os resultados são os mesmos, ou seja, a string é copiada usando essa notação com a mesma precisão de antes. Isso levanta alguns pontos interessantes que discutiremos.

Como os parâmetros são passados por valor, tanto na passagem de um ponteiro de caractere quanto no nome do \textit{array} como acima, o que realmente é passado é o endereço do primeiro elemento de cada \textit{array}. Assim, o valor numérico do parâmetro passado é o mesmo se usarmos um ponteiro de caractere ou um nome de \textit{array} como parâmetro. Isso tenderia a implicar que de alguma forma \textbf{fonte[i]} é o mesmo que \textbf{*(p + i)}.

De fato, isso é verdade, ou seja, sempre que alguém escreve um \textbf{a[i]}, ele pode ser substituído por \textbf{*(a + i)} sem problemas. Na verdade, o compilador criará o mesmo código em ambos os casos. Portanto, vemos que a aritmética de ponteiro é a mesma coisa que a indexação de \textit{array}. Qualquer sintaxe produz o mesmo resultado.

Isso NÃO quer dizer que ponteiros e \textit{arrays} sejam a mesma coisa, eles não são. Estamos apenas dizendo que, para identificar um determinado elemento de um \textit{array}, temos a opção de duas sintaxes, uma usando indexação de \textit{array} e a outra usando aritmética de ponteiros, que produzem resultados idênticos.

Agora, olhando para esta última expressão, parte dela ... \textbf{(a + i)}, é uma adição simples usando o operador \textbf{+} e as regras de C afirmam que tal expressão é comutativa. Ou seja, \textbf{(a + i)} é idêntico a \textbf{(i + a)}. Assim, poderíamos escrever \textbf{*(i + a)} tão facilmente quanto \textbf{*(a + i)}.

Mas \textbf{*(i + a)} poderia ter vindo de \textbf{i[a]}! De tudo isso vem a curiosa verdade que se:
\begin{lstlisting}
	char a[20];
	int i;
\end{lstlisting}
escrever
\begin{lstlisting}
	a[3] = 'x';
\end{lstlisting}
é o mesmo que escrever
\begin{lstlisting}
	3[a] = 'x'
\end{lstlisting}

Tente! Configure um array de caracteres, inteiros ou longos, etc. e atribua ao terceiro ou quarto elemento um valor usando a abordagem convencional e, em seguida, imprima esse valor para ter certeza de que está funcionando. Em seguida, inverta a notação do \textit{array} como fiz acima. Um bom compilador não hesitará e os resultados serão idênticos. Uma curiosidade ... nada mais!

Agora, olhando para a nossa função, quando escrevemos:
\begin{lstlisting}
	dest[i] = fonte[i];
\end{lstlisting}
devido ao fato de que a indexação de \textit{array} e a aritmética de ponteiro produzem resultados idênticos, podemos escrever isso como:
\begin{lstlisting}
	*(dest + i) = *(fonte + i);
\end{lstlisting}

Mas, isso faz 2 adições para cada valor assumido por i. As adições, em geral, levam mais tempo do que as incrementações (como aquelas feitas usando o operador \textbf{++} como em \textbf{i++}). Isso pode não ser verdade em compiladores de otimização modernos, mas nunca se pode ter certeza. Portanto, a versão do ponteiro pode ser um pouco mais rápida do que a versão do \textit{array}.

Outra forma de acelerar a versão do ponteiro seria alterar:
\begin{lstlisting}
	while (*fonte != '\0')
\end{lstlisting}
para simplesmente
\begin{lstlisting}
	while (*fonte)
\end{lstlisting}
já que o valor entre parênteses irá para zero (FALSO) ao mesmo tempo em ambos os casos.

Neste ponto, você pode querer experimentar um pouco escrevendo alguns de seus próprios programas usando ponteiros. Manipulação strings é um bom lugar para experimentar. Você pode querer escrever suas próprias versões de funções padrão como:
\begin{lstlisting}
	strlen();
	strcat();
	strchr();
\end{lstlisting}
e quaisquer outras que você possa ter em seu sistema.

Voltaremos às strings e sua manipulação por meio de ponteiros em um capítulo futuro. Por enquanto, vamos prosseguir e discutir um pouco as estruturas.

