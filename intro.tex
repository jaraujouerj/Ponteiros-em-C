\chapterimage{edisons-electric-pen-1600.jpg} % Chapter heading image

\chapter*{Introdução}
Se você deseja ser proficiente na escrita de código na linguagem de programação C, deve ter um conhecimento prático completo de como usar ponteiros. Infelizmente, os ponteiros C parecem representar um obstáculo para os recém-chegados, particularmente aqueles vindos de outras linguagens de computador, como Python, Java ou Javascript. Muitas dessas linguagens dizem não usar ponteiros, porém, por debaixo dos panos, tudo são ponteiros. A compreensão dos mecanismos dos ponteiros em C permitem um melhor entendimento dos mecanismos e limitações de linguagens como Python e Java, nas quais os ponteiros não são visíveis, mas estão lá, na infraestrutura dessas linguagens, prestando seu serviço.

Para ajudar esses recém-chegados a compreender as dicas, escrevi o seguinte material. Para obter o máximo benefício deste material, considero importante que o usuário seja capaz de executar o código nas várias listagens contidas no artigo. Tentei, portanto, manter todo o código compatível com ANSI para que funcione com qualquer compilador compatível com ANSI. Também tentei bloquear cuidadosamente o código dentro do texto. Dessa forma, com a ajuda de um editor de texto ASCII, você pode copiar um determinado bloco de código para um novo arquivo e compilá-lo em seu sistema. Recomendo que os leitores façam isso, pois ajudará na compreensão do material.

