\chapterimage{beaumont-machine} % Chapter heading image

\chapter*{Prefácio}
Este texto nasceu da necessidade de ensinar ponteiros em C. O tópico de ponteiros sempre foi o grande terror de estudantes iniciais da linguagem de programação C e este tema não era bem esclarecido nos principais livros disponíveis. Apesar do medo dos estudantes, o tema não é complexo, mas exige muita atenção e uma compreensão muito boa sobre o hardware e como funcionam os computadores.

Sempre pensei em escrever um texto como esse, mas esta tarefa sempre foi deixada para depois, até que me deparei com o excelente texto de \textbf{Ted Jensen}. Felizmente ele colocou o texto em domínio público e assim pude aproveitá-lo sem restrições. No início seria apenas uma tradução, porém a última revisão deste texto foi feita há mais de 20 anos. Na época que Ted escreveu este livro e o disponibilizou gratuitamente, a licença \textit{Creative Commons }estava iniciando. Assim, em vez de distribuir como domínio público, escolhi distribuí-la com uma licença mais moderna que mantém o material disponível para todos e pede apenas uma declaração de atribuição.

Livros excelentes se perdem por questões de copyright. Os autores perdem o interesse no tema (ou coisa pior) e bons textos não podem ser atualizados para aproveitar os avanços na computação.

Antes de disponibilizá-lo tentei contato com Ted, mas todas as tentativas foram infrutíferas. O site original não funciona mais e o e-mail não responde. A última atualização do site foi em 2003 e em 2018 ele saiu do ar.

De qualquer modo, mesmo sem conseguir contactá-lo, deixo aqui meu profundo agradecimento a Ted pela sua generosidade em distribuir material de alta qualidade. A maior parte do texto aqui apresentado é minha tradução do texto original de Ted.

Fiz algumas adaptações para modernizar o texto. Na primeira versão do livro existia apenas o C ANSI, depois tivemos algumas pequenas modificações na linguagem que tentei incorporar ao texto original de Ted. 

Chegou um momento que deixou de ser apenas uma tradução, mas uma co-autoria, mas mantenho a gratuidade do texto para todos que queiram utilizá-lo. Assim, se você quiser usar este texto, peço apenas a citação deste material

O arquivo fonte em Latex deste texto pode ser obtido em\\ \url{https://github.com/jaraujouerj/Ponteiros-em-C}\\
e o arquivo em pdf pode ser baixado do meu site pessoal:\\
\url{http://araujo.eng.uerj.br}

\section*{Sobre mim}
Meu nome é João Araujo e sou professor do Departamento de Engenharia de Sistemas e Computação da Faculdade de Engenharia da Universidade do Estado do Rio de Janeiro desde 1990. Trabalho há anos ensinando programação principalmente com a linguagem C. Aprendi C ainda estudante, na UFRJ, nos anos 1980, na versão original do livro K\&R e acompanhei a transição de C por todas as suas versões: ANSI, C99 e C11.

\section*{Uso deste Material}
Tudo aqui contido é liberado sob a licença Creative Commons CC-by-SA. Qualquer pessoa pode copiar ou distribuir este material livremente, desde que siga os parâmetros desta licença. Assim como Ted, `\textit{`a única coisa que peço é que se este material for usado como um auxiliar de ensino em uma aula, eu agradeceria se fosse distribuído na íntegra, ou seja, incluindo todos os capítulos, o prefácio e a introdução. Eu também agradeceria se, em tais circunstâncias, o instrutor de tal classe me deixasse uma nota em um dos endereços abaixo informando-me sobre isso. Escrevi isso com a esperança de que seja útil a outras pessoas e, como não estou pedindo nenhuma remuneração financeira, a única maneira de saber que atingi, pelo menos parcialmente, esse objetivo é por meio do feedback de quem considera este material útil.}'

Instrutor ou não, se este material foi útil para você, me envie uma mensagem em \href{mailto:araujo@eng.uerj.br}{araujo@eng.uerj.br}. Críticas, correções e sugestões também são bem-vindas.

\begin{verbatim}
João Araujo Ribeiro
Departamento de Engenahria de Sistemas e Computação
Faculdade de Engenharia
Universidade do Estado do Rio de Janeiro
Outubro de 2021
\end{verbatim}

\section*{Prefácio original de Ted Jensen}
Este documento tem como objetivo apresentar dicas para programadores iniciantes na linguagem de programação C. Ao longo de vários anos lendo e contribuindo para várias conferências sobre C, incluindo as da \textit{FidoNet} e \textit{UseNet}, observei que um grande número de novatos em C parece ter dificuldade em compreender os fundamentos dos ponteiros. Portanto, empreendi a tarefa de tentar explicá-los em linguagem simples, com muitos exemplos.

A primeira versão deste documento foi colocada em domínio público. Ele foi escolhido por \textit{Bob Stout}, que o incluiu como um arquivo chamado \textbf{PTR-HELP.TXT} em sua coleção amplamente distribuída de \textbf{SNIPPETS}. Desde aquele lançamento original de 1995, adicionei uma quantidade significativa de material e fiz algumas pequenas correções no trabalho original.

Posteriormente, postei uma versão HTML por volta de 1998 em meu site em:\\
http://pweb.netcom.com/~tjensen/ptr/cpoint.htm

Depois de inúmeros pedidos, finalmente saí com esta versão em PDF que é idêntica à versão em HTML citada acima e que pode ser obtida no mesmo site.

\section*{Agradecimentos}
Há tantas pessoas que, sem saber, contribuíram para este trabalho por causa das perguntas que colocaram no \textit{FidoNet C Echo}, ou no \textit{UseNet Newsgroup comp.lang.c}, ou em várias outras conferências em outras redes, que seria impossível listar todas elas. Agradecimentos especiais a \textit{Bob Stout}, que teve a gentileza de incluir a primeira versão deste material em seu arquivo \textbf{SNIPPETS}.

\section*{Sobre o Autor}
Ted Jensen é um engenheiro eletrônico aposentado que trabalhou como designer de hardware ou gerente de designers de hardware na área de gravação magnética. Programar tem sido um hobby seu, intermitentemente, desde 1968, quando ele aprendeu a perfurar cartões para enviá-los a um mainframe. (O mainframe tinha 64K de memória de núcleo magnético!).

\section*{Uso deste Material}
Tudo aqui contido é liberado para o domínio público. Qualquer pessoa pode copiar ou distribuir este material da maneira que desejar. A única coisa que peço é que se este material for usado como um auxiliar de ensino em uma aula, eu agradeceria se fosse distribuído na íntegra, ou seja, incluindo todos os capítulos, o prefácio e a introdução. Eu também agradeceria se, em tais circunstâncias, o instrutor de tal classe me deixasse uma nota em um dos endereços abaixo informando-me sobre isso. Escrevi isso com a esperança de que seja útil a outras pessoas e, como não estou pedindo nenhuma remuneração financeira, a única maneira de saber que atingi, pelo menos parcialmente, esse objetivo é por meio do feedback de quem considera este material útil.

A propósito, você não precisa ser um instrutor ou professor para entrar em contato comigo. Agradeço uma nota de qualquer pessoa que considere o material útil ou que tenha uma crítica construtiva a oferecer. Também estou disposto a responder a perguntas enviadas por e-mail nos endereços abaixo.

\begin{verbatim}
Ted Jensen
Redwood City, California
tjensen@ix.netcom.com
July 1998
\end{verbatim}