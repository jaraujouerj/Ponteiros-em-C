\chapterimage{beaumont-machine} % Chapter heading image

\chapter*{Prefácio}

Este documento tem como objetivo apresentar dicas para programadores iniciantes na linguagem de programação C. Ao longo de vários anos lendo e contribuindo para várias conferências sobre C, incluindo as da \textit{FidoNet} e \textit{UseNet}, observei que um grande número de novatos em C parece ter dificuldade em compreender os fundamentos dos indicadores. Portanto, empreendi a tarefa de tentar explicá-los em linguagem simples, com muitos exemplos.

A primeira versão deste documento foi colocada em domínio público, assim como esta. Ele foi escolhido por \textit{Bob Stout}, que o incluiu como um arquivo chamado \textbf{PTR-HELP.TXT} em sua coleção amplamente distribuída de \textbf{SNIPPETS}. Desde aquele lançamento original de 1995, adicionei uma quantidade significativa de material e fiz algumas pequenas correções no trabalho original.

Posteriormente, postei uma versão HTML por volta de 1998 em meu site em:\\
http://pweb.netcom.com/~tjensen/ptr/cpoint.htm

Depois de inúmeros pedidos, finalmente saí com esta versão em PDF que é idêntica à versão em HTML citada acima e que pode ser obtida no mesmo site.

\section*{Agradecimentos}
Há tantas pessoas que, sem saber, contribuíram para este trabalho por causa das perguntas que colocaram no \textit{FidoNet C Echo}, ou no \textit{UseNet Newsgroup comp.lang.c}, ou em várias outras conferências em outras redes, que seria impossível listar todas elas. Agradecimentos especiais a \textit{Bob Stout}, que teve a gentileza de incluir a primeira versão deste material em seu arquivo \textbf{SNIPPETS}.

\section*{Sobre o Autor}
Ted Jensen é um engenheiro eletrônico aposentado que trabalhou como designer de hardware ou gerente de designers de hardware na área de gravação magnética. Programar tem sido um hobby seu, intermitentemente, desde 1968, quando ele aprendeu a perfurar cartões para enviá-los a um mainframe. (O mainframe tinha 64K de memória de núcleo magnético!).

\section*{Uso deste Material}
Tudo aqui contido é liberado para o domínio público. Qualquer pessoa pode copiar ou distribuir este material da maneira que desejar. A única coisa que peço é que se este material for usado como um auxiliar de ensino em uma aula, eu agradeceria se fosse distribuído na íntegra, ou seja, incluindo todos os capítulos, o prefácio e a introdução. Eu também agradeceria se, em tais circunstâncias, o instrutor de tal classe me deixasse uma nota em um dos endereços abaixo informando-me sobre isso. Escrevi isso com a esperança de que seja útil a outras pessoas e, como não estou pedindo nenhuma remuneração financeira, a única maneira de saber que atingi, pelo menos parcialmente, esse objetivo é por meio do feedback de quem considera este material útil.

A propósito, você não precisa ser um instrutor ou professor para entrar em contato comigo. Agradeço uma nota de qualquer pessoa que considere o material útil ou que tenha uma crítica construtiva a oferecer. Também estou disposto a responder a perguntas enviadas por e-mail nos endereços abaixo.

\begin{verbatim}
Ted Jensen
Redwood City, California
tjensen@ix.netcom.com
July 1998
\end{verbatim}